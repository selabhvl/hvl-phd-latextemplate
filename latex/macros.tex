%====================================================
%-------------Macro definitions go here--------------
%====================================================

%%%%%%%%%%%%%%%%%%
% Shorthand macros
%%%%%%%%%%%%%%%%%%

% references macros in .bib files
\providecommand{\Proc}{Proceedings of the\xspace}
\providecommand{\IntlConf}{International Conference on\xspace}
\providecommand{\EuroConf}{European Conference on\xspace}
\providecommand{\Conf}{Conference on\xspace}
\providecommand{\IntlSymp}{International Symposium\xspace}

% used for code
\newcommand{\code}[1]{{\texttt{#1}}}
% font used first time a new concept is introduced
\newcommand\concept[1]{{\em #1}}
% used when referencing to names in figures
\newcommand{\figitem}[1]{\textbf{\textup{#1}}}
% referencing to functions
\newcommand{\func}[1]{function~\textsc{#1}}
% referencing 
\newcommand{\secref}[1]{Section~\ref{#1}\xspace}
\newcommand{\figref}[1]{Fig.~\ref{#1}\xspace}
\newcommand{\tabref}[1]{Table~\ref{#1}\xspace}
\newcommand{\algoref}[1]{Algorithm~\ref{#1}\xspace}
\newcommand{\equref}[1]{Equation~\ref{#1}\xspace}
\newcommand{\defref}[1]{Definition~\ref{#1}\xspace}
\newcommand{\listref}[1]{Listing~\ref{#1}\xspace}
\newcommand{\lineref}[1]{Line~\ref{#1}\xspace} % the line reference for code in listing 
\newcommand{\linesref}[2]{Lines~\ref{#1}-\ref{#2}\xspace} 
\newcommand{\partref}[1]{Part~\ref{#1}\xspace}
\newcommand{\partsref}[1]{Parts~\ref{#1}\xspace}
\newcommand{\chapref}[1]{Chapter~\ref{#1}\xspace}
\newcommand{\eref}[1]{~(\ref{#1})}
\renewcommand{\equiv}[0]{\ensuremath{:=}}
\newcommand{\paperheader}[2]{\noindent\textbf{Paper #1}: \textit{#2}\\}
\newcommand{\paperitem}[3]{\noindent\textbf{Paper #1}: \textit{#2}\vspace{1em}\\\noindent #3\vspace{2em}}
\newcommand{\tfinal}{\ensuremath{T_{\text{f}}}}
\newcommand{\papernum}[1]{\textbf{#1}}
\newcommand{\etal}{et\,al.}
\newcommand{\cf}{cf.\xspace}
\newcommand{\eg}{e.\,g., }
\newcommand{\ie}{i.\,e., }

%
% used to comment out things
%
\newcommand\ignore[1]{}

%
% Differentials
%
\newcommand{\tdiff}[2]{\ensuremath{\frac{d#2}{d{#1}}}}
\newcommand{\tdifforder}[3]{\ensuremath{\frac{d^{#2}#3}{d{#1}^{#2}}}}
\newcommand{\pdiff}[2]{\ensuremath{\frac{\partial#2 }{\partial#1}}}
\newcommand{\pdifforder}[3]{\ensuremath{\frac{\partial^{#2}#3}{\partial{#1}^{#2}}}}

%
% Linear algebra
%
\renewcommand{\vec}[1]{\ensuremath{\mathbf{#1}}}
\newcommand{\mat}[1]{\ensuremath{\mathbf{#1}}}
\newcommand{\tildemat}[1]{\ensuremath{\widetilde{\mat{#1}}}}

